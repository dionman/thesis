\section{Summary \& discussion} %\& Future Work}

In this chapter we have shown that the mobility networks of individuals exhibit significant diversity, and the topology of the mobility network itself, without explicit privacy-revealing labels, may be unique and therefore uniquely identifying.

An individual's mobility network is dynamic over time.
Therefore, an adversary with access to mobility data of a person from one time period cannot simply test for graph isomorphism to retrieve the same user from a dataset recorded at a different point in time.
Hence we proposed graph kernel methods to detect structural similarities between two mobility networks, and thus provide the adversary with information on the likelihood that two mobility networks represent the same individual.
While graph kernel methods are imperfect predictors, they perform significantly better than a random strategy and therefore our approach induces significant privacy loss.
Our approach does not make use of geographic information or fine-grained temporal information. Therefore, our method is immune to commonly adopted privacy intending practices of geographic information masking or removal, and temporal cloaking, and thus it may lead to new mobility deanonymization attacks.

Moreover, we find that reducing the number of edges~(transitions between locations) in a mobility network does not necessarily make the network more privacy-preserving, while user anonymity is violated even when reducing the number of nodes~(locations).
Conversely, releasing the frequency of node visits and the direction of transitions in a mobility network does aid the identifiablility of a mobility network for adversaries applying graph kernel similarity metrics on identified historical data.
We provide empirical evidence that neighborhood relations in the high-dimensional spaces generated by the tested deep graph kernels remain meaningful for our dataset of networks~\citep{Beyer}.
Further work is needed to shed more light on the geometry of those spaces in order to derive the optimal substructures and dimensionality required to support best graph set matching.
More work is also required to understand the sensitivity of our approach to the time period over which mobility networks are constructed.
There is also an opportunity to explore better ways of exploiting pairwise distance information.

Beyond emphasizing the vulnerability of popular anonymization techniques based on user-specific location pseudonymization, our work provides insights into network features that can facilitate the identifiability of location traces.
Our framework also opens the door to new anonymization techniques that can apply structural similarity methods to individual traces in order to cluster people with similar mobility behaviour.
This approach may then support statistically faithful population mobility studies on mobility networks securing $k-$anonymity guarantees for participants.

Apart from graph kernel similarity metrics, tools for network deanonymization can also be sought in the direction of graph mining: applying heavy subgraph mining techniques~\citep{Bogdanov2011}, or searching for persistent cascades~\citep{Morse16}.
Frequent substructure pattern mining (e.g.~gSpan,~\textcitecustom{Yan2002}) and discriminative frequent subgraph mining (e.g.~CORK,~\textcitecustom{Thoma2010}) techniques can also be considered.

Our methodology is, in principle, applicable to all types of data where individuals transition amongst a set of discrete states.
Therefore, the performance of such retrieval strategies can also be evaluated on different categories of datasets, such as web browsing histories, or smartphone application usage sequences.

A drawback of our current approach is that it cannot be directly used to mimic individual or group mobility by synthesizing traces.
Fitting a generative model on mobility traces and then defining a kernel on this model~\citep{song11} may provide better anonymity, and therefore privacy, and it would also support the generation of artificial traces which mimic the mobility of users.

{
Standard Bayesian inference algorithms are prohibitively expensive in the
regime of modern large-scale data. Recent work has found that
a small, weighted subset of data (a \emph{coreset}) may be used in place
of the full dataset during inference, taking advantage of data redundancy to
reduce computational cost. However, this approach has limitations 
in the increasingly common setting of sensitive, high-dimensional data. 
Indeed, we prove that there are situations in which 
the Kullback-Leibler~(KL) divergence between the \emph{optimal} coreset 
and the true posterior grows with data dimension; and as coresets include
a subset of the original data, they cannot be constructed in a manner
that preserves individual privacy.
We address both of these issues with a single unified solution, \emph{Bayesian
pseudocoresets}---a small weighted collection of synthetic
``pseudodata''---along with a variational optimization method to select both
pseudodata and weights.  The use of pseudodata (as opposed to
the original datapoints) enables both the summarization of high-dimensional data
and the  differentially private summarization of
sensitive data. Real and
synthetic experiments on high-dimensional data demonstrate that Bayesian 
pseudocoresets achieve significant improvements in posterior approximation error compared to
traditional coresets, and that pseudocoresets provide privacy without
a significant loss in approximation quality. 
}

This chapter is based on~\citep{psvi}.
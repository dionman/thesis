In~\cref{chap:chap2}, we exposed the prohibitive computational limitations of Bayesian inference in the regime of modern large-scale data, and discussed coreset-based summarization as a viable solution for scalable approximate inference under statistical guarantees. In~\cref{chap:chap3}, we considered a case study on a massive high-dimensional dataset capturing longitudinal mobility information of a population, and quantified the privacy loss incurred via coarse representations of the datapoints that can be used for fast data analysis.
Motivated by the quest for scalable learning methods on sensitive data, in this chapter we propose \emph{pseudocoreset  variational inference}, a general-purpose approximate inference method designed to enable scalable inference on high-dimensional datasets, under the guarantees of approximate differential privacy.

We begin by investigating the shortcomings of existing Bayesian coreset constructions 
in the increasingly common setting of sensitive, high-dimensional data. 
In particular, we prove that there are situations in which 
the Kullback-Leibler divergence between the \emph{optimal} coreset 
and the true posterior grows with data dimension; and as coresets include
a subset of the original data, they cannot be constructed in a manner
that preserves individual privacy.
We address both of these issues with a single unified solution, \emph{Bayesian
pseudocoresets}---a small weighted collection of synthetic
``pseudodata''---along with a variational optimization method to select both
pseudodata and weights.  The use of pseudodata (as opposed to
the original datapoints) enables both the summarization of high-dimensional data
and the  differentially private summarization of
sensitive data. Real and
synthetic experiments on high-dimensional data demonstrate that Bayesian 
pseudocoresets achieve significant improvements in posterior approximation error compared to
traditional coresets, and that pseudocoresets provide privacy without
a significant loss in approximation quality. 


%This chapter is based on~\citep{psvi}.
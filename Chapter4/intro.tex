\section{Introduction}
\label{sec:intro}


Large-scale data---which has become the norm in many scientific and commercial applications of statistical machine learning---creates
an inherently difficult setting for the modern data analyst. Exploring such data is difficult because it cannot all be
obtained and directly visualized at once; one is typically limited to accessing potentially nonrepresentative random subsets of data.
Exploring models is similarly hard, as training even a single model can be a computationally expensive, slow, and unreliable process.
And as many sources of large-scale data contain sensitive information about 
individuals (e.g., electronic health records and social network data),
these challenges are coupled with growing privacy concerns  that preclude direct access to individual datapoints completely. 

Large-scale data does offer one reprieve to the analyst: it often exhibits a significant degree of redundancy. Most data are 
not unique or particularly informative for modeling and exploration. Based on this notion, data summarization methods have been developed  
that provide the practitioner with a compressed---but still statistically representative---version of the large dataset for analysis.
Summarizations have been developed for a variety of purposes, e.g., reducing the cost of computing with kernel matrices via Nystr{\"o}m-type approximations~\citep{drineas05,musco17,agrawal19} or sparse pseudo-input parameterizations~\citep{snelson05},
Bayesian inference \citep{huggins16,huggins17,campbell18,campbell19jmlr}, maximum likelihood 
parameter estimation \citep{dumouchel99,madigan02}, 
linear regression \citep{zhou08,guhaniyogi15},
geometric shape approximation \citep{agarwal05},
clustering \citep{feldman11,lucic16,bachem15,braverman16}, and dimensionality reduction \citep{feldman16}.


A common form of summarization is that of a sparse, weighted subset of the original dataset---a \emph{coreset} \citep{agarwal05}. 
Coresets have two distinct advantages over other possible summarization modalities: they are easily interpreted, and can often be used
as the input to standard data analysis algorithms without modification. 
But as the dimensionality of a dataset grows, its constituent 
datapoints tend to become more ``unique'' and cannot represent one another well. Indeed, in the context of Bayesian inference---the focus of the
present work---we show that the \emph{optimal} coreset posterior approximation to the true posterior has KL divergence that scales with the dimension 
of the data in a simple problem setting (\cref{prop:original_coreset_fails}). 
Furthermore, directly releasing a subset of the original data precludes any possibility of
individual privacy under the current standard 
of differential privacy \citep{dwork2006calibrating,dwork14}. Past work
addresses this issue in the context of clustering and computational geometry \citep{feldman09,feldman17}---with
the remarkable property that the privatized coreset may be queried \emph{ad infinitum} without
loss of privacy---but no such method exists for Bayesian posterior inference.

In this work, we develop a novel technique for data summarization in the context of Bayesian inference under the constraints that the method 
is scalable and easy to use, creates an intuitive summarization, applies to high-dimensional data, and enables privacy control.
Inspired by past work \citep{madigan02,zhou08,snelson05}, instead of using constituent datapoints, we use synthetic \emph{pseudodata} to summarize the large dataset, resulting in a \emph{pseudocoreset}. We show that in the high-dimensional problem setting of \cref{prop:original_coreset_fails}, 
the optimal pseudocoreset with just one pseudodata point recovers the exact posterior, a significant improvement upon the optimal standard coreset of any size.
As in past work on Bayesian coresets \citep{campbell19neurips}, we formulate pseudocoreset construction as variational inference, 
and provide a stochastic optimization method~(\cref{sec:pseudocoresets}). As a consequence of the use of pseudodata---as well as privacy-preserving 
stochastic gradient descent mechanisms~\citep{abadi16,park16, jalko17}---we show that our method can easily be modified to output
a privatized pseudocoreset. The chapter concludes with experimental results demonstrating the performance of pseudocoresets on 
real and synthetic data~(\cref{sec:experiments}).

This chapter is based on~\citep{psvi}.

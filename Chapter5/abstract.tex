{
	In~\cref{chap:chap4}, we proposed a new Bayesian coreset construction that addresses scalability, dimensionality and privacy preservation requirements in large-scale inference. In this chapter, we design one more coreset construction that aims to resolve another frequently occurring challenge in probabilistic inference over real-world datasets: that of model misspecification.
	
	Modern machine learning applications should be able to address the intrinsic challenges arising over inference on massive real-world datasets, including scalability and robustness to outliers. Despite the multiple benefits of Bayesian methods (such as uncertainty-aware predictions, incorporation of experts knowledge, and hierarchical modeling), the quality of classic Bayesian inference depends critically on whether observations conform with the assumed data generating model, which is impossible to guarantee in practice. In this chapter, we propose a variational inference method that, in a principled way, can simultaneously scale to large datasets, and robustify the inferred posterior with respect to the existence of outliers in the observed data. Reformulating Bayes theorem via the \bdiv, we posit a robustified pseudo-Bayesian posterior as the target of inference. Moreover, relying on the recent formulations of Riemannian coresets for scalable Bayesian inference, we propose a sparse variational approximation of the robustified posterior and an efficient stochastic black-box algorithm to construct it. Overall our method allows releasing cleansed data summaries  that can be applied broadly in scenarios including structured data corruption. We illustrate the applicability of our approach in diverse simulated and real datasets, and various statistical models, including Gaussian mean inference, logistic and neural linear regression, demonstrating its superiority to existing Bayesian summarization methods in the presence of outliers. %Furthermore, our dataset reduction scheme enables us to efficiently compute data Shapley value approximations, making a decisive step towards equitable valuation of individual information in massive scale data marketplaces.
}

%This chapter is based on~\citep{beta-cores}.
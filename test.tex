\documentclass[a4paper,12pt,index,custombib]{Classes/PhDThesisPSnPDF} 

\DeclareOption{tchypersetup}{\hypersetup{colorlinks=true, allcolors=SubtleColor, pdfborder={0 0 0}}}
\DeclareOption{blackhypersetup}{\hypersetup{colorlinks=false, allcolors=black, pdfborder={0 0 0}}}

%autonum
\newif\ifuseautonum
\useautonumfalse
\DeclareOption{autonum}{\useautonumtrue}

\ProcessOptions

\ifuseautonum \usepackage{autonum} \fi

\DeclareCiteCommand{\cite}
{\usebibmacro{prenote}}
{\usebibmacro{citeindex}%
	\printtext[bibhyperref]{\usebibmacro{cite}}}
{\multicitedelim}
{\usebibmacro{postnote}}

\DeclareCiteCommand*{\cite}
{\usebibmacro{prenote}}
{\usebibmacro{citeindex}%
	\printtext[bibhyperref]{\usebibmacro{citeyear}}}
{\multicitedelim}
{\usebibmacro{postnote}}


\DeclareCiteCommand{\parencite}[\mkbibparens]
{\usebibmacro{prenote}}
{\usebibmacro{citeindex}%
	\printtext[bibhyperref]{\usebibmacro{cite}}}
{\multicitedelim}
{\usebibmacro{postnote}}

\DeclareCiteCommand*{\parencite}[\mkbibparens]
{\usebibmacro{prenote}}
{\usebibmacro{citeindex}%
	\printtext[bibhyperref]{\usebibmacro{citeyear}}}
{\multicitedelim}
{\usebibmacro{postnote}}




\newcommand{\textcitec}[1]{\textcolor{mydarkblue}{\textcite{#1}}}

\RequirePackage[unicode=true]{hyperref}

\usepackage[dvipsnames]{xcolor}
\definecolor{mydarkblue}{rgb}{0,0.08,0.45}

\if@print
% For Print version
\hypersetup{
	final=true,
	plainpages=false,
	pdfstartview=FitV,
	pdftoolbar=true,
	pdfmenubar=true,
	bookmarksopen=true,
	bookmarksnumbered=true,
	breaklinks=true,
	linktocpage,
	colorlinks=true,
	linkcolor=black,
	urlcolor=black,
	citecolor=black,
	anchorcolor=black
}
\ifsetCustomMargin
% Margin to be define in preamble using geometry package
\else
\RequirePackage[paper=\PHD@papersize,hmarginratio=1:1,
vmarginratio=1:1,scale=0.75,bindingoffset=5mm]{geometry}
\fi

\if@twoside
\hypersetup{pdfpagelayout=TwoPageRight}
\else
\hypersetup{pdfpagelayout=OneColumn}
\fi

\else
% For PDF Online version
\hypersetup{
	final=true,
	plainpages=false,
	pdfstartview=FitV,
	pdftoolbar=true,
	pdfmenubar=true,
	bookmarksopen=true,
	bookmarksnumbered=true,
	breaklinks=true,
	linktocpage,
	colorlinks=true,
	linkcolor=mydarkblue,
	urlcolor=mydarkblue,
	citecolor=mydarkblue,
	anchorcolor=mydarkblue
}



\begin{document}

\begingroup
Large-scale data does offer one reprieve to the analyst: it often exhibits a significant degree of redundancy. Most datapoints are 
not unique or particularly informative for modeling and exploration. Based on this notion, data summarization methods have been developed  
that provide the practitioner with a compressed---but still statistically representative---version of the large dataset for analysis.
Summarizations have been developed for a variety of purposes, e.g., reducing the cost of computing with kernel matrices via Nystr{\"o}m-type approximations~\citep{drineas05,musco17,agrawal19} or sparse pseudo-input parameterizations for Gaussian processes~\citep{williams01,csato02,snelson05,titsias09},
Bayesian inference~\parencite{huggins16,huggins17,campbell18,campbell19jmlr}, maximum likelihood 
parameter estimation~\citep{dumouchel99,madigan02}, 
linear regression~\citep{zhou08,guhaniyogi15},
geometric shape approximation \citep{agarwal05},
clustering \citep{feldman11,lucic16,bachem15,braverman16}, and dimensionality reduction \citep{feldman16}.
\end{document}          
